%% haskell-exts-cs.tex
%
% Copyright 2018  Rudy Matela
%
% This text is available under (at your option):
%   * Creative Commons Attribution-ShareAlike 3.0 Licence
%   * GNU Free Documentation License version 1.3 or Later
%

\documentclass{refcard}
\usepackage[T1]{fontenc}
\usepackage{rotating}
\usepackage{amssymb}

\renewcommand{\familydefault}{\sfdefault}
\newcommand{\la}{\textbackslash}


\title{Haskell Yesod Cheat Sheet}

\cright{
	Copyright 2018, Rudy Matela --
	Compiled on \today{} \\
	Upstream: \texttt{https://github.com/rudymatela/concise-cheat-sheets}
}{
	This text is available under
	the Creative Commons Attribution-ShareAlike 3.0 Licence, \\
	\textbf{or} (at your option), the GNU Free Documentation License version 1.3 or Later.
}
\version[~\\]{0.0}


\begin{document}

\maketitle

\section{File Extensions}

\begin{tabular}{Cl}
	.hs          & Haskell source \\
	.lhs         & Literate Haskell source \\
	.cabal       & CABAL Build System config \\[1ex]
	.html (.htm) & Hypertext Markup Language \\
	.css         & Cascading Style Sheet \\
	.js          & Javascript \\[1ex]
	.hamlet      & HTML template language \\
	.lucius      & Curly CSS template language \\
	.cassius     & Indenty CSS template language \\
	.julius      & Javascript with variable interpolation \\[1ex]
	.md          & Markdown markup language \\
	.yaml (.yml) & YAML Ain't Markup Language (data / configs) \\
\end{tabular}

\section{Packages}

\subsection{Hackage / Stackage}

\begin{tabular}{Cl}
	yesod-bin & Provides the `yesod` executable. \\
\end{tabular}

\subsection{Ubuntu}

\subsection{Arch Linux}

\section{Default Directory Structure}

\begin{tabular}{Cl}
	\I{myprj}.cabal     & modules / deps. / build options \\
	stack.yaml          & stackage dependencies \\
	app/                & \\
	~~main.hs           & entry point for the production server \\
	~~devel.hs          & entry point for \C{~yesod devel~}\\
	config/             & \\
	config/settings.yml & settings: host, port, logging, dbs... \\
	config/routes       & all application routes \\
	config/models       & models: Haskell records / DB Tables \\
	src/                & Haskell source files \\
	src/Application.hs  & imported by \C{main.hs} and \C{devel.hs} \\
	src/Handler/        & Request handlers \\
	src/Handler/Home.hs & Home route handler \\
	static/             & static files (\C{<URL>/static/}) \\
	templates/          & template files: hamlet, lucius, cassius... \\
	\multicolumn{2}{C}{templates/default-layout.hamlet} \\
	\multicolumn{2}{C}{templates/default-layout-wrapper.hamlet} \\
	\multicolumn{2}{C}{templates/default-layout.lucius} \\
	templates/homepage.hamlet & homepage HTML template \\
	templates/homepage.lucius & homepage CSS template \\
	test/               & test files \\
	test/...            & TODO: describe me \\
\end{tabular}

\section{Miscellaneous Commands}

\begin{ldesc}
\li[List available project templates] stack templates | grep yesod
\li[Scaffold (minimal) new project]   stack new \I{myprj} yesod-minimal
\li[Scaffold (simple) new project]    stack new \I{myprj} yesod-simple
\li[Scaffold (sqlite) new project]    stack new \I{myprj} yesod-sqlite
\li[Scaffold (postgres) new project]  stack new \I{myprj} yesod-postgres
\li[Prints Yesod's version]           yesod version
\li[Adds a new handler]               yesod add-handler
\li[Runs development web server]      yesod devel
\li[Default dev.~server locations]    http://localhost:3000/ \li
                                      https://localhost:3443/
\end{ldesc}


\section{Routes}

\section{Models}

\end{document}
